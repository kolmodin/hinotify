\begin{hcarentry}{hinotify}
\report{Lennart Kolmodin}
\status{alive}
\makeheader

``hinotify'' is a simple Haskell wrapper for the Linux kernel's inotify
mechanism. inotify allows applications to watch file changes since
Linux kernel 2.6.13.  You can for example use it to do a proper
locking procedure on a set of files, or keep your application up do
date on a directory of files in a fast and clean way.

As file and directory notification is available for many operating systems
upcoming work will include to try to find a common API that could be shared
for all platforms. Last work has been to see what's possible to do under
Microsoft Windows, and finding a suiting API for both platforms. This has
been a joint work with Niklas Broberg. We're still looking for contributors
to *BSD and Mac OS X. If you're interested, contact us.

\FurtherReading
\begin{compactitem}
\item Development version:

  \texttt{darcs get}

  \url{http://www.haskell.org/~kolmodin/code/hinotify/}
\item Latest released version:

  \url{http://www.haskell.org/~kolmodin/code/hinotify/download/}
\item Documentation:

  \url{http://www.haskell.org/~kolmodin/code/hinotify/docs/api}
\item inotify:

  \url{http://www.kernel.org/pub/linux/kernel/people/rml/inotify/}
\end{compactitem}
\end{hcarentry}
